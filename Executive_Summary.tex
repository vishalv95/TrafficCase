\documentclass[7pt]{article}
\title{Executive Summary}
\author{Rahul Gupta, Phoebe Lin, Srija Nalla, and Vishal Vusirikala}
\date{\today}
\usepackage[margin=.5in]{geometry}
\begin{document}
\maketitle
\section{Overview}
\subsection{Traffic Concerns}
Austin's burgeoning population has experienced a growth rate of 2.5\% year over year, placing the city in the top five fastest growing populations in the country. This rapid increase in Austin's populace has direct impacts on the city's four major expressways (Loop 1, SH 71, I-35, Route 183) in the form of exacerbated traffic jams, increasingly frequent accidents, and longer travel times. 

\subsection{Congestion Pricing}
To offset these effects on rush hour traffic, we propose a congestion pricing model. Similar to initiatives in Los Angeles, London, and New York City, this model dynamically associates a price with the use of "express lanes." Unlike existing models, our system will factor in real-time traffic levels to determine the toll price, as opposed to a price set solely based on the time of day. We hypothesize that using real time traffic data facilitates a more flexible, long-term solution.
 
\subsection{Incentivizing Ride Sharing}
To reduce overall traffic, we provide a platform and financial incentives to facilitate ride sharing. Our platform builds upon the existing CapMetro mobile application, which touts a user base of around 200 thousand Austin residents. Similar to existing ride sharing services like Uber, this app integration allows drivers to post requests for passengers seeking to travel to proximate destinations. \\\\
Commuters that carpool on any of the aforementioned expressways enjoy a subsidized fare that scales with the number of passengers. Passengers additionally have the option to split the fee among members of their party through the CapMetro app. This strategy greatly cuts down on rush hour traffic by grouping commuters that are willing and able to travel together to offset the higher rush hour fees.

\section{Technical Specifications}
\subsection{Toll Lane Logistics}
\subsubsection{Lane Expansion}
Congestion pricing can be coupled with existing or proposed initiatives to expand toll lanes on the expressways. Our model suggests an adaptive method for establishing prices with these toll lanes. The proposals for each freeway are discussed below:

\begin{center}
    \begin{tabular}{ | l | l | l | p{5cm} |}
    \hline
    Expressway & Daily Traffic Volume & Cost & Existing Proposal Details \\ \hline
    Loop 1 & 130,000 vehicles & \$200 million & 1 express lane in each direction for 11 miles and a tunnel under MoPac to divert downtown traffic. \\ \hline
    SH 71 & 58,000 vehicles & \$149 million & 3.9 mile limited access toll road for airport access. \\ \hline
    I-35 & 220,000 vehicles & \$4.3 billion & Additional toll lane for each direction. 10 year proposal, still requires approval.\\ \hline
    Route 183 & 60,000 vehicles & \$740 million & 3 new toll lanes and up to 3 non-tolled lanes for 183 South. Additional toll lanes and 4 flyovers for 183 North. \\ \hline 
	\end{tabular}
\end{center}

\subsubsection{Motion Sensors}
Our proposal collects vehicular speed data using motion sensor technology. We ascertain traffic volume as a function of the average vehicle speed, and apply our dynamic pricing model appropriately.

\subsubsection{TxTag Integration}
Texas currently uses three tag systems to identify cars: TxTag (Austin), EZ Tag (Houston), and TollTag (DFW). Interoperability of all these systems creates an infrastructural foundation for implementing our congestion pricing plan. 
\subsection{Application Development}
\subsubsection{Location Services}
CapMetro's mobile app uses location services (GPS) to determine a user's current position with a high degree of precision. Current GPS technology enables phones to find other nearby devices, and this feature helps developers find and group users who are sharing rides. Carpoolers who wish to claim the ride sharing discount must confirm through the app. 
\subsubsection{Recommender Systems}
Passengers have the ability to select a driver from a list of suggested drivers generated based on proximity of destinations, estimated arrival times, user ratings of the driver, and degrees of separation from the user's social network through Facebook integration. Drivers enjoy the benefit of no toll charge to offset the additional cost of gas and have a similar list of passengers that they can request generated by the same criteria.
\section{Financials}

\center \textbf{Revenue} \\
\begin{center}
	\begin{tabular}{ | l | l | l | l |}	\hline
    & Daily Revenue & Annualized Revenue & 2025 Annual Revenue Projection \\ \hline
    Loop 1 & \$191,841 & \$57,109,000 & \$69,615,552 \\ \hline
    SH 71 & \$92,984 & \$27,802,216 & \$33,890,746 \\ \hline
    I-35 & \$305,997 & \$91,493,103 & \$111,529,582 \\ \hline
    Route 183 & \$94,214 & \$28,169,986 & \$34,339,056 \\ \hline 
    Totals & \$685,036  & \$204,574,305 & \$249,374,936 \\ \hline
	\end{tabular}
\end{center}

\center \textbf{Motion Sensor Costs} \\ 
\begin{center}
	\begin{tabular}{ | l | l | l | l |}	\hline
    & Upfront Year (2016) & 2017 - 2020 Annual Cost & 2021-2025 Annual Cost \\ \hline
    Loop 1 & \$2,450,000& \$106,963 & \$161,277 \\ \hline
    SH 71 & \$1,320,000 & \$52,695 & \$79,453 \\ \hline
    I-35 & \$4,250,000 & \$210,779 & \$317,811 \\ \hline
    Route 183 & \$2,560,000 & \$26,741 & \$40,319 \\ \hline 
    Totals & \$10,580,000 & \$397,177 & \$598,860 \\ \hline
	\end{tabular}

\end{center}

\center \textbf{App Development Costs} \\
\begin{center}
	\begin{tabular}{ | l | l | l | l |}	\hline
    & Upfront Year (2016) & 2017 - 2020 Annual Cost & 2021-2025 Annual Cost \\ \hline
    Development & \$35,000 & \$4,500& \$2,060\\ \hline
    Marketing & \$80,000 & \$26,375 & \$6,162 \\ \hline
    Maintenance & \$26,000 & \$26,990 & \$28,862 \\ \hline
    Data Management & \$670,000 & \$264,500 & \$68,400 \\ \hline 
    Totals & \$811,000 & \$322,365 & \$105,484 \\ \hline
	\end{tabular}
\end{center}


\end{document}